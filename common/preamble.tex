% --- Packages
\usepackage{setspace}    % setstretch command to provide extra line spacing
\usepackage{xspace,fbox}
\usepackage{luaquotes}   % automatic typographical quotes, and '' in texttt
\usepackage{xifthen}     % used in 'oneline' macro in this file
\usepackage{graphicx}    % used in 'oneline' macro in this file [resizebox?]
\usepackage{emoji}
\usepackage{awesomebox}  % \notebox, \warningbox
\usepackage{tkz-fct}
\usepackage{import}
\usepackage{tabto}
\usepackage[export]{adjustbox}     % allow for 'right' option in \includegraphics
\usepackage[hang,flushmargin]{footmisc}
\usepackage{hyperref}

% NOT included: bm for boldface mathematical text. That is for legacy fonts.
% If I am using Tex Gyre Pagella Math, for instance, then that is a modern OTF font.
% The unicode-math package provides \symbf{...} for bold text, and symit, stmsf, symbfit, symsfbf
% (according to ChatGPT).
\newcommand{\bm}[1]{\symbf{#1}} 

\hypersetup{
    colorlinks,
    linkcolor={red!50!black},
    citecolor={blue!50!black},
    urlcolor={blue!80!black}
}

% --- Margin and page style
\pagestyle{empty}
\setlrmarginsandblock{2.5cm}{2.5cm}{*}
\setulmarginsandblock{2.5cm}{*}{1}
\checkandfixthelayout

% --- Command to squeeze some text down to one line.
%     Good for when a sentence _just_ goes over.
%     Source: https://gist.github.com/malloc47/5298181
\newcommand{\oneline}[1]{%
  \newdimen{\namewidth}%
  \setlength{\namewidth}{\widthof{#1}}%
  \ifthenelse{\lengthtest{\namewidth < \textwidth}}%
  {#1}% do nothing if shorter than text width
  {\resizebox{\textwidth}{!}{#1}}% scale down
}

% --- Convenience commands -------------------------------------
\DeclareMathOperator{\cis}{cis}
% \DeclareMathOperator{\gcd}{gcd}
\DeclareMathOperator{\LCM}{lcm}
\renewcommand{\Re}{\operatorname{Re}}
\renewcommand{\Im}{\operatorname{Im}}
\newcommand{\ds}{\displaystyle}

\newcommand{\percent}{\%}

% --- \pt{a} gives (a); \pt{b} gives (b); etc. If it's not (a), there will be space before.
%     This is meant for writing hints.
%     TODO: consider making this a macro in LBT. It's risky to use a name like 'pt'.
\newcommand{\pt}[1]{\textbf{\ifstrequal{#1}{a}{(#1)~}{\enspace (#1)~}}}

% --- Quoting and quadding.
\newcommand{\q}[1]{`#1'}
\newcommand{\qq}[1]{``#1''}
\newcommand{\qqquad}{\quad\quad\quad}
\newcommand{\qqqquad}{\quad\quad\quad\quad}

% --- Brackets
\newcommand{\br}[1]{\left( #1 \right)}
\newcommand{\sqbr}[1]{\left[ #1 \right]}
\newcommand{\sqbrtext}[1]{\[ #1 \]}

% --- Fractions
\newcommand{\half}{\tfrac12}
\newcommand{\third}{\tfrac13}
\newcommand{\quarter}{\tfrac14}
\newcommand{\twothirds}{\tfrac23}
\newcommand{\threequarters}{\tfrac34}

% --- Fractions involving pi
\newcommand{\pitwo}{\frac \pi 2}
\newcommand{\pithree}{\frac \pi 3}
\newcommand{\pifour}{\frac \pi 4}
\newcommand{\pifive}{\frac \pi 5}
\newcommand{\pisix}{\frac \pi 6}
\newcommand{\piseven}{\frac \pi 7}
\newcommand{\pieight}{\frac \pi 8}

\newcommand{\pitwothree}{\frac {2\pi} 3}
\newcommand{\pithreefour}{\frac {3\pi} 4}

% Reciprocal fractions (shorthand for 1 over ...)
\newcommand{\recip}{\frac1}

% --- Derivatives
\newcommand{\deriv}[2][x]{\frac{d}{d#1} #2}
\newcommand{\ddx}{\tfrac{d}{dx}}
\newcommand{\derivative}[2]{\frac{d#1}{d#2}}
\newcommand{\secondderivative}[2]{\frac{d^2 #1}{d#2^2}}

% --- Set
\newcommand{\set}[1]{\left\{ #1 \right\}}

% --- Absolute value
\newcommand{\abs}[1]{\left| #1 \right|}

% --- Units
\newcommand{\m}{\,\unit{m}}
\newcommand{\mps}{\,\unit{m.s^{-1}}}
\newcommand{\mpsps}{\,\unit{m.s^{-2}}}
\newcommand{\kg}{\,\unit{kg}}
\newcommand{\km}{\,\unit{km}}
\newcommand{\kmh}{\,\unit{km.h^{-1}}}

% --- Conjugates (bar)
\newcommand{\zbar}{\ensuremath{\overline{z}}}
\newcommand{\wbar}{\ensuremath{\overline{w}}}

% --- e to the i pi on ...
\newcommand{\eipi}[2][]{e^{i\frac{#1\pi}{#2}}}
\newcommand{\eitheta}{\ensuremath{e^{i\theta}}}

% --- Inverse sin, cos, tan
\newcommand{\invsin}{\sin^{-1}}
\newcommand{\invcos}{\cos^{-1}}
\newcommand{\invtan}{\tan^{-1}}

% --- cos theta etc
\newcommand{\sintheta}{\sin\theta}
\newcommand{\costheta}{\cos\theta}
\newcommand{\tantheta}{\tan\theta}

% --- ln with || or ()
\newcommand{\lnabs}[1]{\ln\abs{#1}}
\newcommand{\lnbr}[1]{\ln\br{#1}}

% --- sets: naturals, integers, reals, ...
\newcommand{\bbN}{\mathbb{N}}
\newcommand{\bbZ}{\mathbb{Z}}
\newcommand{\bbQ}{\mathbb{Q}}
\newcommand{\bbR}{\mathbb{R}}
\newcommand{\bbC}{\mathbb{C}}
