\begin{lbt}
  !DEBUG
  [@META]
    TEMPLATE lbt.Doc.Chapter
    TITLE Core commands in lbt.Basic
    LABEL ch.basic
    SOURCES LbtDoc

  [+BODY]
    TEXT The \package{lbt.Basic} template implements:
    ITEMIZE .o compact
    :: document divisions (part, chapter, section, subsection, subsubsection, paragraph, subparagraph)
    :: low-level typesetting facilities (vertical space, arbitrary commands and environments, flushleft, flushright, center)
    :: things you expect in normal Latex editing, whether built in or using a common plugin (lists, tables, columns, verbatim, various math environments)
    :: things that are generally useful (include PDF pages, three levels of headings, place two items side-by-side)

    % ------------------------------------------------------------ Document divisions
    SECTION Document divisions

    TEXT All the Latex commands are present. We present \code{CHAPTER} and \code{SECTION} in \cref{ex:basic-divisions}, but don't show the typeset results as we don't want to affect the chapter of this book!

    LBTEXAMPLE .o vertical, output = 1
    :: (caption) Chapters, sections and plain text paragraphs
    :: (label) ex:basic-divisions
    :: .v <<
      CHAPTER (label) ch:canopy :: Beneath the canopy

      TEXT When we first visited the rainforest, \dots

      SECTION (label) sec:life-leaves :: Life through the leaves

      TEXT From the forest floor, you can't see any direct light at all \dots
    >>

    % --------------------------------------- Ordinary text and whitespace
    SECTION Ordinary text and whitespace

    TEXT \Cref{ex:basic-text} shows the \code{TEXT} command, which outputs one or more paragraphs. \code{TEXT*} suppresses the (final) \verb|\par|. You can input vertical space with \code{VSPACE}. Further, all commands accept optional arguments \code{pre} and \code{post} for including some surrounding whitespace.

    LBTEXAMPLE .o vertical
    :: (caption) \code{TEXT} (single and multiple paragraphs) and \code{VSPACE}
    :: (label) ex:basic-text
    :: .v <<
      TEXT Can you guess which author wrote these sentences about chess?
      VSPACE 1em
      TEXT
      :: The chessboard is the world; the pieces are the phenomena of the universe; the rules of the game are what we call the laws of Nature.
      :: The beauty of a move lies not in its appearance but in the thought behind it.
      :: In chess, as in life, a man is his own most dangerous opponent.
      TEXT* .o pre = 2em :: It was\dots
      TEXT an early world champion.
    >>

    TEXT Other low-level formatting commands include \code{CLEARPAGE} and \code{VFILL}, both of which are hard to demonstrate, but predictable. Finally, there is \code{VSTRETCH}, which helps spread items out vertically. For example, \cref{ex:basic-vstretch} spreads out three (very short) paragraphs to fill a page, and allocates more whitespace in the middle than the other two.

    LBTEXAMPLE .o float, output = 0
    :: (caption) Using \code{VSTRETCH} to spread out paragraphs on a page
    :: (label) ex:basic-vstretch
    :: .v <<
      TEXT Top paragraph.
      VSTRETCH 1
      TEXT Middle paragraph.
      VSTRETCH 1.5
      TEXT Bottom paragraph.
      VSTRETCH 1
      CLEARPAGE
    >>

    TEXT The \code{CLEARPAGE} is necessary to ensure that all vertical space on the page is used.

    % ----------------------------------------------- Itemized and enumerated lists
    SECTION Itemized and enumerated lists

    TEXT The paragraphs in \cref{ex:basic-text} would be better written as a list. \cref{ex:basic-list-1} shows the chess quotes in an itemized list. \cref{ex:basic-list-2} enumerates some principles of chess opening theory.

    LBTEXAMPLE .o vertical
    :: (caption) An itemized list
    :: (label) ex:basic-list-1
    :: .v <<
      TEXT Emanual Lasker wrote some punchy sentences in his \emph{Manual of Chess} (1925):
      ITEMIZE
      :: The chessboard is the world; the pieces are the phenomena of the universe; the rules of the game are what we call the laws of Nature.
      :: The beauty of a move lies not in its appearance but in the thought behind it.
      :: In chess, as in life, a man is his own most dangerous opponent.
    >>

    LBTEXAMPLE .o vertical
    :: (caption) An enumerated list
    :: (label) ex:basic-list-2
    :: .v <<
      TEXT An experienced player has three key objectives in the opening.
      ENUMERATE
      :: Focus pawns and/or pieces on the central four squares.
      :: Activate the minor pieces.
      :: Castle.
    >>

    TEXT
    :: \code{ITEMIZE} and \code{ENUMERATE} use the \package{enumitem} package in the backgound. You can use the kwarg \code{spec} to pass options through to the underlying \code{itemize} or \code{enumerate} environment. Further, the oparg \code{compact} provides an easy way to tighten the list, as \cref{ex:basic-list-3} demonstrates.
    :: You can also use opargs \code{notop} and \code{sep} to control vertical space in a more specific but still convenient way. See the documentation.
    :: Finally, using \package{enumitem}'s \code{newlist} and \code{setlist} commands, you can define your own list style with the formatting you require. Suppose you now have a \code{shoppinglist} environment. You can make use of that with the \code{env} oparg. This is demonstrated in \cref{ex:basic-list-custom}

    LBTEXAMPLE .o vertical
    :: (caption) A compact enumerated list with custom label
    :: (label) ex:basic-list-3
    :: .v <<
      TEXT An experienced player has three key objectives in the opening.
      ENUMERATE .o compact :: (spec) (1)
      :: Focus pawns and/or pieces on the central four squares.
      :: Activate the minor pieces.
      :: Castle.
    >>

    LBTEXAMPLE .o vertical
    :: (caption) A custom shopping list
    :: (label) ex:basic-list-custom
    :: .v <<
      CMD newlist :: shoppinglist :: itemize :: 1
      CMD setlist :: [shoppinglist]
      :: label=\ding{111}, left=4pt, itemsep = 2pt, topsep = 2pt

      TEXT Things to buy on the way home:
      ITEMIZE .o env = shoppinglist
      :: oat milk
      :: tofu
      :: vegan cheese
    >>

    % ----------------------------------------------- Description lists
    SECTION Description lists

    NOTE These are not implemented yet. Watch this space.

    % ----------------------------------------------- Automatic multi-level lists
    SECTION Automatic multi-level lists

    TEXT
    :: If you want to typeset a multi-level list using standard Latex environments, you will end up with a lot of boilerplate. The \lbtlogo{} command \code{LIST} offers great convenience, as \cref{ex:basic-list-multi} demonstrates.
    :: The markers chosen are not a great fit for the list content, but they show some of the possibilities. The \code{49} refers to the dingbats provided by the \package{pifont} package. See the documentation for more details.

    LBTEXAMPLE .o horizontal
    :: (caption) A mutli-level list
    :: (label) ex:basic-list-multi
    :: .v <<
      LIST .o markers = circnum * 49
      :: Fruits
      :: * Citrus
      :: * * Orange
      :: * * Lemon
      :: * * Lime
      :: * Berries
      :: * * Strawberry
      :: * * Blueberry
      :: Vegetables
      :: * Leafy greens
      :: * * Spinach
      :: * * Kale
      :: * Root vegetables
      :: * * Carrot
      :: * * Beetroot
      :: * * Potato
    >>

    % ----------------------------------------------- Tables
    SECTION Tables

    STO fn1 :: 1 :: If you really want \lbtlogo{} to support \package{tabularx} or something else, feel free to request it. Of course, you can implement it yourself, too.
    TEXT
    :: \lbtlogo{} makes an opinionated choice that tables are best set using \package{tabularray}.\footnote{◊fn1} The \code{TABLE} command is a light layer over a \code{tblr} environment. You provide all the specifications (details of column alignment and cell formatting) in the mandatory \code{spec} kwarg.
    :: \cref{ex:basic-table-1} shows left, center and right column alignment, and bold top-row headings, and a horizontal line between the headings and the data. This is a very simple table.
    :: \cref{ex:basic-table-2} improves on the above by using the \package{booktabs} library to gain access to \latexcmd{toprule}, \latexcmd{midrule} and \latexcmd{bottomrule}. It also introduces some padding to the third column, and increases the row separation.
    :: Most tables created with Latex are for displaying information in articles or books. But tables can be used for other purposes, such as educational handouts. \cref{ex:basic-table-3} demonstrates the use of space (setting column widths and row heights) and background colour. It also puts most cells in math mode, and shows all borders.
    :: It takes a bit of digging in the \package{tabularray} manual to find the right incantations to make all this happen, but ultimately it is easier to achieve the desired results with this package than with any other. And if you need multiple tables with the same format, it is easy to create your own table type with \latexcmd{NewTblrEnviron}. If you created an \code{invoice} table, for instance, you could invoke it with \code{TABLE .o invoice}.

    NOTE No attempt has been made so far to create tables with spanning cells. Whether any specific support from \lbtlogo{} is required remains to be seen.

    NOTE The thing about \code{TABLE .o invoice} is a bald-faced lie. That needs to be implemented. (Easy, though.)

    SUBSECTION Loading table data from a file

    TEXT The \code{TABLE} command makes it easy to load CSV or TSV data from a file and use it as the rows of the table. \cref{ex:basic-table-dataload} demonstrates this with cumulative normal distribution values, which are clearly better located in a data file than in a Latex file.

    LBTEXAMPLE .o vertical
    :: (caption) Simple table example
    :: (label) ex:basic-table-1
    :: .v <<
      TABLE .o center :: (spec) colspec = {l c r},
        » row{1}={font=\bfseries}
      :: Author & Year of Birth & Published Works
      :: \hline
      :: William Shakespeare & 1564 & 39
      :: Jane Austen         & 1775 & 7
      :: Charles Dickens     & 1812 & 20
      :: Leo Tolstoy         & 1828 & 48
    >>

    LBTEXAMPLE .o vertical
    :: (caption) Simple table example with more formatting
    :: (label) ex:basic-table-2
    :: .v <<
      TABLE .o center :: (spec) colspec = {l c r},
      » row{1}={font=\bfseries}, cell{2-Z}{3}={appto=\hspace*{3em}}, rowsep=3pt
      :: \toprule
      :: Author & Year of Birth & Published Works
      :: \midrule
      :: William Shakespeare & 1564 & 39
      :: Jane Austen         & 1775 & 7
      :: Charles Dickens     & 1812 & 20
      :: Leo Tolstoy         & 1828 & 48
      :: \bottomrule
    >>

    LBTEXAMPLE .o vertical
    :: (caption) Table with spacing, color, math mode and lines
    :: (label) ex:basic-table-3
    :: .v <<
      TABLE .o center :: (spec) colspec = {cccc}, hlines, vlines, row{2-Z}={7mm, mode=math},
      » column{1-2}={2cm}, column{3-4}={3cm}, row{1}={bg=Turquoise!35, font=\bfseries}
      :: Sum & Product
      :: 11 & 18 & 9+2=11 & 9\times2=18
      :: 21 & 110 & 10+11=21 & 10\times11=110
      :: 12 & 36
      :: 18 & 77
      :: 16 & 48
      :: 13 & 40
    >>

    LBTEXAMPLE .o vertical, scale = 0.9
    :: (caption) Table with data loaded from a file
    :: (label) ex:basic-table-dataload
    :: .v <<
      TABLE .o centre
      :: (spec) colspec={rrrrrrrrrrr}, hlines, vlines,
      »    column{1}={bg=gray9, font=\bfseries}, row{1} = {bg=gray9, font=\itshape},
      »    cell{1}{1} = {mode=math}
      :: (datafile) media/standard-normal-cumulative.tsv
      :: @datarows 1
      :: @datarows 22..27
    >>





    % ----------------------------------------------- Margins and justification
    SECTION Margins and justification

    TEXT \cref{ex:basic-margins} demonstrates \code{INDENT}, which adjusts the left and right margin using the \code{adjustwidth} environment from the \code{changepage} package. The first argument defines the inward margin for left and right margins. (Note that the right margin adjustment defaults to zero.) The second argument is the text.

    LBTEXAMPLE .o vertical
    :: (caption) Using \code{INDENT} to adjust left and (optionally) right margins
    :: (label) ex:basic-margins
    :: .v <<
      INDENT 4cm, 2cm :: It is a truth universally acknowledged, that a single man in possession of a good fortune, must be in want of a wife.

      INDENT 6cm :: However little known the feelings or views of such a man may be on his first entering a neighbourhood, this truth is so well fixed in the minds of the surrounding families, that he is considered as the rightful property of some one or other of their daughters.
    >>

    TEXT \cref{ex:basic-justification} demonstrates \code{FLUSHLEFT}, \code{FLUSHRIGHT} and \code{CENTER}, which affect the justification of the contained paragraph(s). Incidentally, it also demonstrates \code{STO}, which saves (\qq{stores}) some content for a limited amount of time. For more information about \code{STO}, see \cref{ex:sto}.

    LBTEXAMPLE .o vertical
    :: (caption) Left, center and right justification
    :: (label) ex:basic-justification
    :: .v <<
      STO text :: 3 :: However little known the feelings or views of such a man may be on his first entering a neighbourhood, this truth is so well fixed in the minds of the surrounding families, that he is considered as the rightful property of some one or other of their daughters.

      FLUSHLEFT ◊text
      CENTER .o pre = 1ex :: ◊text
      FLUSHRIGHT .o pre = 1ex :: ◊text
    >>

    STO sep :: 1 :: {:}{:}
    NOTE The examples show that \code{INDENT} and \code{CENTER}, et cetera, operate only on the paragraph(s) given to them. This raises a question: how do you center, say, a whole block of \lbtlogo{} code? \par There are two answers. First, you could manually invoke the \code{center} environment with \code{BEGIN center}, then place your code, then \code{END center}. This is low-level and not in the spirit of \lbtlogo. The second option is to put all your content in a register with \code{STO .o lbt ◊sep content ◊sep .v << ... >>}, and then use \code{CENTER ◊content}. \par \lbtTodo{For this documentation, readers should be directed to a section that demonstrates STO in detail.}

\end{lbt}
