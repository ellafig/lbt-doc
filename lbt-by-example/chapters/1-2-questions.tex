
\begin{lbt}
  [@META]
    TEMPLATE  lbt.Doc.Chapter
    TITLE     Worksheet or exam questions with lbt.Questions
    LABEL     ch.questions
    SOURCES   LbtDoc, lbt.Questions

  [+BODY]

    TEXT The \package{lbt.Questions} template offers useful commands for typesetting questions, subquestions, and multiple-choice options.

    TEXT Use \code{Q} for a top-level question and \code{QQ} for a question part (see \cref{ex:questions}).

    LBTEXAMPLE .o horizontal
    :: (caption) Question parts and subparts
    :: (label) ex:questions
    :: .v <<
      Q Name three different kinds of clouds.

      Q Evaluate the following.
      QQ $3 + 12 / 4$
      QQ $(3 + 12) / 4$
    >>

    TEXT \cref{ex:q-hor} shows the use of \code{QQ*} to lay out questions parts horizontally.

    LBTEXAMPLE .o vertical
    :: (caption) Arranging question parts horizontally
    :: (label) ex:q-hor
    :: .v <<
      Q How many vowels appear in each word?
      QQ* [ncols=3]
      :: appear :: Augustine :: crimson :: toast :: glyph :: transformer
    >>

    TEXT Use \code{MC} or \code{MC*} to lay out \textbf{multiple-choice options} (see \cref{ex:q-mc}).

    LBTEXAMPLE .o vertical
    :: (caption) Multiple choice answers
    :: (label) ex:q-mc
    :: .v <<
      Q Which planet of the solar system has the most moons?
      MC Earth :: Mars :: Jupiter :: Saturn

      Q Which planet of the solar system has the fewest moons?
      MC* [ncols=4] :: Mercury :: Venus :: Uranus :: Neptune
    >>

    TEXT A question can have a \textbf{source} and/or a \textbf{note} preceding the text, and you can change the colour of the question marker (see \cref{ex:q-source-notes}).

    LBTEXAMPLE .o vertical
    :: (caption) Question source and notes
    :: (label) ex:q-source-notes
    :: .v <<
      Q .o color=purple :: (source) HSC 2005 :: (note) sigma notation
      :: Evaluate $\displaystyle \sum_{n=3}^5 (2n+1)$.
    >>

    TEXT If you want \emph{all} questions in your document to have a purple marker, you can include the line \texttt{OPTIONS  Q.color = purple} in the \texttt{[@META]} part of your \lbtlogo{} environment.

\end{lbt}
