\begin{lbt}
  !DRAFT
  [@META]
    TEMPLATE lbt.Doc.Chapter
    TITLE Mathematical text \textsf{(lbt.Math)}
    LABEL ch.math
    SOURCES LbtDoc, lbt.Math

  [+BODY]
    TEXT The \package{lbt.Math} template provides several affordances for typing mathematical text.
    ITEMIZE .o compact
    :: The \code{simplemath} macro is a replacement for the inline and display math environments \Verb|$ ... $| and \Verb|$$ .. $$| (or their Latex equivalents). It allows you to type mathematical text more succinctly and with fewer backslashes.
    :: The \code{integal} macro simplifies typing definite and indefinite (single) integrals.
    :: The \code{vector} and \code{vectorijk} macro greatly simplify typing vectors.
    :: The \code{MATH} command gives you easy access to \package{amsmath}'s collection of environments: align, gather, multiline, \dots
    :: A collection of macros like \code{mathlistand}, \code{mathsum}, \code{mathseqdots} and several others provide a convenient way to type mathematical text like \qq{\mathlistand{a,b,c}} or \qq{\mathseqdots{y,1,2,3,n}}.

    % ------------------------------------------------------------ The simplemath macro
    SECTION The \code{simplemath} macro

    BOX :: \verb|\lbtDefineMacros{sm = lbt.Math:simplemath}|

    TEXT
    :: Typing mathematics in Latex involves a lot of backslashes. With some preprocessing, which is easier in Lua than in Tex, we can improve this and make the resulting document text easier to read and write. Additionally, abbreviations are defined for Greek letters.
    :: \cref{ex:math-sm-1} demonstrates three keywords (\code{sin}, \code{cos}, \code{equiv})  and two abbreviations \code{al} and \code{th} (for \verb|\alpha| and \verb|\theta|).
    :: \cref{ex:math-sm-2} demonstrates several keywords: \code{forall}, \code{exists}, \code{in}, \code{implies} and \code{text}.
    :: \cref{ex:math-sm-3-comparison} shows the Latex and simplemath code for the previous examples, allowing you to see a comparison.

    LBTEXAMPLE .o vertical
    :: (caption) simplemath (inline) with \Verb{\sm{...}}
    :: (label) ex:math-sm-1
    :: .v <<
      ITEMIZE .o compact
      :: \sm{cos th = 0.27}
      :: \sm{cos^2 al + sin^2 al equiv 1}
    >>

    LBTEXAMPLE .o vertical
    :: (caption) simplemath (display) with \Verb{\sm{ ... }} (note spaces)
    :: (label) ex:math-sm-2
    :: .v <<
      TEXT The \emph{intermediate value theorem}: \sm{ forall f text{ continuous on } [a,b], f(a) < 0 < f(b) implies exists c in (a,b) text{ such that } f(c) = 0 }
    >>

    LBTEXAMPLE .o vertical, output = 0
    :: (caption) Comparison of Latex and simplemath
    :: (label) ex:math-sm-3-comparison
    :: .v <<
      \sm{cos th = 0.27}              $\cos\theta = 0.27$

      \sm{cos^2 al + sin^2 al equiv 1}     $\cos^2\alpha + \sin^2\alpha \equiv 1$

      \sm{ forall f text{ continuous on } [a,b], f(a) < 0 < f(b) implies exists c in (a,b) text{ such that } f(c) = 0 }
      \[ \forall f \text{ continuous on } [a,b], f(a) < 0 f(b) \Longrightarrow \exists c \in (a,b) \text{ such that } f(c = 0) \]
    >>







    % ------------------------------------------------------------ The integral macro
    SECTION The \code{integral} macro

    % ------------------------------------------------------------ The vector macro and friends
    SECTION The \code{vector} macro and friends

    % ------------------------------------------------------------ The MATH command
    SECTION The \code{MATH} command

    % ------------------------------------------------------------ The mathlistand macro and friends
    SECTION The \code{mathlistand} macro and friends

\end{lbt}
