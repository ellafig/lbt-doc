\documentclass[12pt,oneside]{memoir}

\title{LBT by Example}
\author{Gavin Sinclair}
\date{May 2025}

% [[[ Paper size and layout ---------------------------------------------------------
% Page size: a gentle vertical rectangle for comfortable screen viewing
\setstocksize{270mm}{210mm}            % Slightly shorter than A4
\settrimmedsize{270mm}{210mm}{*}       % Center the trimmed page

% Typeblock: a generous text area that avoids long line lengths
\settypeblocksize{220mm}{140mm}{*}     % Height, width, vertical centering
\setlrmargins{*}{*}{1.2}               % Left-right margins, symmetrical
\setulmargins{*}{*}{1.2}               % Top-bottom margins

% Header and footer space
\setheadfoot{14pt}{20pt}               % Room for headers and footers
\setheaderspaces{*}{*}{1.2}            % Balanced spacing

% Finalise the layout
\checkandfixthelayout
% ]]]

% [[[ LBT settings and macro definitions --------------------------------------------
\usepackage{lbt}
  \lbtSettings{
    SettingsFile = .lbtSettings,
  }
  \lbtLoadTemplates{PWD/../lbt-templates}
  \lbtGlobalOpargs{
    vector.format = bold,
    LBTEXAMPLE.shrinkmargin = 3em,
    LBTEXAMPLE.float = true,
    LBTEXAMPLE.position = tbp,
    EXAMPLETABLE.shrinkmargin = 3em,
  }
  % \lbtDefineLatexMacros{V = lbt.Math:vector, sm = lbt.Math:simplemath}
  \input ../common/lbt_macros.tex
% ]]]

  % [[[ Packages and settings that directly affect appearance -------------------------
\usepackage[svgnames]{xcolor}
  \definecolor{commandcolor}{RGB}{20, 40, 100}
  \definecolor{captioncolor}{RGB}{40, 60, 120}
  \definecolor{lbtTodo}{RGB}{200, 30, 30}

\renewcommand*{\cftpartpresnum}{Part~}
\renewcommand*{\cftpartnumwidth}{4em}  % adjust width as needed
\renewcommand*{\cftpartfont}{\color{Cerulean}\bfseries}
\renewcommand*{\cftpartpagefont}{\color{Cerulean}\bfseries}

\usepackage{float}
  \newfloat{example}{p}{loe}[chapter]
  \floatname{example}{Example}

\usepackage{caption}[2022/02/20]
  \captionsetup[example]{
    font = {small, sf, color=captioncolor},
    labelfont = bf,
    labelsep = quad,
    justification = raggedleft,
    singlelinecheck = false,
    position = below,
    skip = 4pt,
  }
  \captionsetup[table]{
    font = {small, sf, color=captioncolor},
    labelfont = bf,
    labelsep = quad,
    justification = centering,
    singlelinecheck = false,
    position = above,
    skip = 12pt,
  }

% TODO: implement or remove these
% \RecustomVerbatimCommand{\Verb}{Verb}{formatcom=\color{commandcolor}}
% \renewcommand{\FancyVerbFormatInline}[1]{\Verb[formatcom=\color{commandcolor}]}

\usepackage{fontspec, unicode-math}
  %% % Serif font: fbb (https://ctan.org/pkg/fbb)
  %% \setmainfont[%
  %% UprightFeatures={StylisticSet=01},
  %% BoldFeatures={StylisticSet=01},
  %% ]{fbb}
  % Serif font: Libertinus Serif
\setmainfont{Tex Gyre Termes}[Scale=1.083]
  % Math font: Libertinus math (https://github.com/alif-type/libertinus)
  \setmathfont{Libertinus Math}    % removed option AutoFakeBold
  % Sans-serif font: Gillius ADF (http://arkandis.tuxfamily.org/adffonts.html)
  \setsansfont{GilliusADF}         % removed option Numbers=OldStyle
  % Monospaced font: Inconsolata (https://www.ctan.org/pkg/inconsolata)
  \setmonofont{DejaVu Sans Mono}[Scale=0.9]
% ]]]

% [[[ General preamble --------------------------------------------------------------

% Sans-serif headings NOTE: these don't work, but see memoir manuaul p. 81
% \setpartheadstyle{\sffamily\Huge\bfseries}
% \setchapterheadstyle{\sffamily\huge\bfseries}
% \setsecheadstyle{\sffamily\Large\bfseries}
% \setsubsecheadstyle{\sffamily\large\bfseries}
% \setsubsubsecheadstyle{\sffamily\normalsize\bfseries}
% \setparagraphheadstyle{\sffamily\normalsize\bfseries}
% \setsubparagraphheadstyle{\sffamily\normalsize\bfseries}

% Language and spacing
\usepackage[english]{babel}
\usepackage{microtype}
\usepackage{luaquotes}

% Chapter style
\chapterstyle{hangnum}  % or 'default', 'veelo', 'dash', 'section' etc.

% Fancy headers (built-in)
\pagestyle{ruled}  % or 'plain', 'headings', 'companion', etc.

% Frames for examples
\usepackage{tcolorbox}

% Better control of floats
\usepackage{float}
\usepackage{placeins}

% Hyperlinks
\usepackage[hidelinks]{hyperref}

% References to examples
\usepackage{cleveref}
  \crefformat{example}{\textsf{\small\bfseries\color{captioncolor}#2Example~#1#3}}
  \Crefformat{example}{\textsf{\small\bfseries\color{captioncolor}Example~#2#1#3}}
  \crefformat{table}{\textsf{\small\bfseries\color{captioncolor}#2Table~#1#3}}
  \Crefformat{table}{\textsf{\small\bfseries\color{captioncolor}Table~#2#1#3}}

% ]]]

% [[[ Document commands ------------------------------------------------------------
\newcommand{\package}[1]{{\color{commandcolor}\textsf{#1}}}
\newcommand{\code}[1]{{\color{commandcolor}\texttt{#1}}}
\newcommand{\boldcode}[1]{{\bfseries\color{commandcolor}\texttt{#1}}}
\newcommand{\lbtlogo}{\textsc{lbt}}
\newcommand{\qq}[1]{``#1''}
\newcommand{\q}[1]{`#1'}
\newcommand{\lbtTodo}[1]{{\color{lbtTodo}\textbf{TODO:} #1}}
\newcommand{\lbtNote}[1]{{\color{lbtTodo}\textbf{NB:} #1}}
\newcommand{\latexcmd}[1]{\code{\textbackslash{}#1}}
% ]]]

% [[[ Necessities for examples -----------------------------------------------------
\UseTblrLibrary{booktabs}
\UseTblrLibrary{siunitx}
% ]]]

\begin{document} % -----------------------------------------------------------------

\frontmatter  % Roman numbering (i, ii, ...) for TOC, preface, etc.

% -------------------------------------------------- Title page
\begin{titlingpage}
  \centering
  \vspace*{3cm}
  {\Huge LBT by Example\par}
  \vspace{1cm}
  {\large Gavin Sinclair\par}
  \vfill
  {\large May 2025\par}
\end{titlingpage}

% -------------------------------------------------- Contents and examples
% \begingroup
%   \let\addcontentsline\relax
%   \tableofcontents
% \endgroup
\tableofcontents*
\listof{example}{List of Examples}

% -------------------------------------------------- Preface and Introduction
\begin{lbt}
  !DEBUG
  [@META]
    TEMPLATE lbt.Doc.Chapter
    TITLE Introduction
    LABEL ch.introduction

  [+BODY]
    TEXT The Preface will give some introductory remarks about \lbtlogo.
  
\end{lbt}

\begin{lbt}
  !DEBUG
  [@META]
    TEMPLATE lbt.Doc.Chapter
    TITLE Preface
    LABEL ch.preface

  [+BODY]
    TEXT The Introduction will give a brief but somewhat comprehensive example.
  
\end{lbt}



% -------------------------------------------------- Main matter
\mainmatter   % Arabic numbering (1, 2, ...) begins

\part{Core templates}

\begin{lbt}
  !DEBUG
  [@META]
    TEMPLATE lbt.Doc.Chapter
    TITLE Core commands
    LABEL ch.basic
    SOURCES LbtDoc

  [+BODY]
    TEXT The \package{lbt.Basic} template implements:
    ITEMIZE .o compact
    :: document divisions (part, chapter, section, subsection, subsubsection, paragraph, subparagraph)
    :: low-level typesetting facilities (vertical space, arbitrary commands and environments, flushleft, flushright, center)
    :: things you expect in normal Latex editing, whether built in or using a common plugin (columns, verbatim, various math environments)
    :: things that are generally useful (include PDF pages, three levels of headings, place two items side-by-side)

    TEXT It also implements lists and tables, which are demonstrated in \cref{ch:liststables}.

    % ------------------------------------------------------------ Document divisions
    SECTION Document divisions

    TEXT All the Latex commands are present. We present \code{CHAPTER} and \code{SECTION} in \cref{ex:basic-divisions}, but don't show the typeset results as we don't want to affect the chapter of this book!

    LBTEXAMPLE .o vertical, output = 1
    :: (caption) Chapters, sections and plain text paragraphs
    :: (label) ex:basic-divisions
    :: .v <<
      CHAPTER (label) ch:canopy :: Beneath the canopy

      TEXT When we first visited the rainforest, \dots

      SECTION (label) sec:life-leaves :: Life through the leaves

      TEXT From the forest floor, you can't see any direct light at all \dots
    >>

    % --------------------------------------- Ordinary text and whitespace
    SECTION Ordinary text and whitespace

    TEXT \Cref{ex:basic-text} shows the \code{TEXT} command, which outputs one or more paragraphs. \code{TEXT*} suppresses the (final) \verb|\par|. You can input vertical space with \code{VSPACE}. Further, all commands accept optional arguments \code{pre} and \code{post} for including some surrounding whitespace.

    LBTEXAMPLE .o vertical
    :: (caption) \code{TEXT} (single and multiple paragraphs) and \code{VSPACE}
    :: (label) ex:basic-text
    :: .v <<
      TEXT Can you guess which author wrote these sentences about chess?
      VSPACE 1em
      TEXT
      :: The chessboard is the world; the pieces are the phenomena of the universe; the rules of the game are what we call the laws of Nature.
      :: The beauty of a move lies not in its appearance but in the thought behind it.
      :: In chess, as in life, a man is his own most dangerous opponent.
      TEXT* .o pre = 2em :: It was\dots
      TEXT an early world champion.
    >>

    TEXT Other low-level formatting commands include \code{CLEARPAGE} and \code{VFILL}, both of which are hard to demonstrate, but predictable. Finally, there is \code{VSTRETCH}, which helps spread items out vertically. For example, \cref{ex:basic-vstretch} spreads out three (very short) paragraphs to fill a page, and allocates more whitespace in the middle than the other two.

    LBTEXAMPLE .o float, output = 0
    :: (caption) Using \code{VSTRETCH} to spread out paragraphs on a page
    :: (label) ex:basic-vstretch
    :: .v <<
      TEXT Top paragraph.
      VSTRETCH 1
      TEXT Middle paragraph.
      VSTRETCH 1.5
      TEXT Bottom paragraph.
      VSTRETCH 1
      CLEARPAGE
    >>

    TEXT The \code{CLEARPAGE} is necessary to ensure that all vertical space on the page is used.





    % ----------------------------------------------- Margins and justification
    SECTION Margins and justification

    TEXT \cref{ex:basic-margins} demonstrates \code{INDENT}, which adjusts the left and right margin using the \code{adjustwidth} environment from the \code{changepage} package. The first argument defines the inward margin for left and right margins. (Note that the right margin adjustment defaults to zero.) The second argument is the text.

    LBTEXAMPLE .o vertical
    :: (caption) Using \code{INDENT} to adjust left and (optionally) right margins
    :: (label) ex:basic-margins
    :: .v <<
      INDENT 4cm, 2cm :: It is a truth universally acknowledged, that a single man in possession of a good fortune, must be in want of a wife.

      INDENT 6cm :: However little known the feelings or views of such a man may be on his first entering a neighbourhood, this truth is so well fixed in the minds of the surrounding families, that he is considered as the rightful property of some one or other of their daughters.
    >>

    TEXT \cref{ex:basic-justification} demonstrates \code{FLUSHLEFT}, \code{FLUSHRIGHT} and \code{CENTER}, which affect the justification of the contained paragraph(s). Incidentally, it also demonstrates \code{STO}, which saves (\qq{stores}) some content for a limited amount of time. For more information about \code{STO}, see \cref{ex:sto}.

    LBTEXAMPLE .o vertical
    :: (caption) Left, center and right justification
    :: (label) ex:basic-justification
    :: .v <<
      STO text :: 3 :: However little known the feelings or views of such a man may be on his first entering a neighbourhood, this truth is so well fixed in the minds of the surrounding families, that he is considered as the rightful property of some one or other of their daughters.

      FLUSHLEFT ◊text
      CENTER .o pre = 1ex :: ◊text
      FLUSHRIGHT .o pre = 1ex :: ◊text
    >>

    STO sep :: 1 :: {:}{:}
    NOTE The examples show that \code{INDENT} and \code{CENTER}, et cetera, operate only on the paragraph(s) given to them. This raises a question: how do you center, say, a whole block of \lbtlogo{} code? \par There are two answers. First, you could manually invoke the \code{center} environment with \code{BEGIN center}, then place your code, then \code{END center}. This is low-level and not in the spirit of \lbtlogo. The second option is to put all your content in a register with \code{STO .o lbt ◊sep content ◊sep .v << ... >>}, and then use \code{CENTER ◊content}. \par \lbtTodo{For this documentation, readers should be directed to a section that demonstrates STO in detail.}


    % ----------------------------------------------- Other stuff...
    SECTION Other stuff\dots

    TEXT This is a placeholder.

    % ----------------------------------------------- Passing content through to Latex
    SECTION Passing content through to Latex

    TEXT
    :: The \code{LATEX} command is simple: it allows you to pass plain Latex code through to the compiler. This is already achievable with the \code{TEXT} or \code{TEXT*} command, but the name \code{LATEX} better represents the intention.
    :: Recall\footnote{\lbtNote{This is not actually documented anywhere yet}} that \code{CMD} exists for single commands, so cases like \code{CMD bigskip} are already taken care of. Assuming, then, that you want to pass something more complicated through to Latex, the idiomatic way is to use the \code{.v << ... >>} verbatim block, which allows you to include newlines in your code.
    :: \cref{ex:basic-latex-math} demonstrates typesetting some complex mathematics among some text. Note that \code{MATH} could handle the mathematics just fine, but the example serves to show the purpose of \code{LATEX}.

    LBTEXAMPLE .o vertical
    :: (caption) Passing Latex code through with \code{LATEX}
    :: (label) ex:basic-latex-math
    :: (substitute) XX/>>
    :: .v <<
      TEXT We now compute the definite integral of $f(x)$ from $-1$ to $2\pi$:
      LATEX .v <<
        \begin{align*}
        \int_{-1}^{2\pi} f(x)\,dx 
        &= \int_{-1}^{0} x^2\,dx 
          + \int_{0}^{\pi} \sin x\,dx 
          + \int_{\pi}^{2\pi} 1\,dx \\
        &= \left[\frac{x^3}{3}\right]_{-1}^{0} 
          + \left[-\cos x\right]_{0}^{\pi} 
          + \left[x\right]_{\pi}^{2\pi} \\
        &= \left(0 - \left(-\frac{1}{3}\right)\right) 
          + \left(-\cos \pi + \cos 0\right) 
          + \left(2\pi - \pi\right) \\
        &= \frac{1}{3} + (1 + 1) + \pi \\
        &= \frac{1}{3} + 2 + \pi
        \end{align*}
      XX
    >>



    


\end{lbt}

\begin{lbt}
  !DEBUG
  [@META]
    TEMPLATE lbt.Doc.Chapter
    TITLE Lists and tables
    LABEL ch.liststables
    SOURCES LbtDoc

  [+BODY]
    TEXT The \package{lbt.Basic} template also implements commands related to lists and tables. The list commands make use of the \package{enumitem} package, and the table command uses \package{tabularray}. The commands are:
    ITEMIZE .o compact
    :: \code{ITEMIZE} for bulleted lists;
    :: \code{ENUMERATE} for numbered lists;
    :: \code{LIST} for multi-level lists (bulleted and/or numbered);
    :: \code{TABLE} for tables

    TEXT Tables can be inline (default) or floating, in which case you can specify a label, caption and position. Provision is made for loading the data in a table from a file.

    % ----------------------------------------------- Itemized and enumerated lists
    SECTION Itemized and enumerated lists

    TEXT The paragraphs in \cref{ex:basic-text} would be better written as a list. \cref{ex:listtable-list-1} shows the chess quotes in an itemized list. \cref{ex:listtable-list-2} enumerates some principles of chess opening theory.

    LBTEXAMPLE .o vertical
    :: (caption) An itemized list
    :: (label) ex:listtable-list-1
    :: .v <<
      TEXT Emanual Lasker wrote some punchy sentences in his \emph{Manual of Chess} (1925):
      ITEMIZE
      :: The chessboard is the world; the pieces are the phenomena of the universe; the rules of the game are what we call the laws of Nature.
      :: The beauty of a move lies not in its appearance but in the thought behind it.
      :: In chess, as in life, a man is his own most dangerous opponent.
    >>

    LBTEXAMPLE .o vertical
    :: (caption) An enumerated list
    :: (label) ex:listtable-list-2
    :: .v <<
      TEXT An experienced player has three key objectives in the opening.
      ENUMERATE
      :: Focus pawns and/or pieces on the central four squares.
      :: Activate the minor pieces.
      :: Castle.
    >>

    TEXT
    :: \code{ITEMIZE} and \code{ENUMERATE} use the \package{enumitem} package in the backgound. You can use the kwarg \code{spec} to pass options through to the underlying \code{itemize} or \code{enumerate} environment. Further, the oparg \code{compact} provides an easy way to tighten the list, as \cref{ex:listtable-list-3} demonstrates.
    :: You can also use opargs \code{notop} and \code{sep} to control vertical space in a more specific but still convenient way. See the documentation.
    :: Finally, using \package{enumitem}'s \code{newlist} and \code{setlist} commands, you can define your own list style with the formatting you require. Suppose you now have a \code{shoppinglist} environment. You can make use of that with the \code{env} oparg. This is demonstrated in \cref{ex:listtable-list-custom}

    LBTEXAMPLE .o vertical
    :: (caption) A compact enumerated list with custom label
    :: (label) ex:listtable-list-3
    :: .v <<
      TEXT An experienced player has three key objectives in the opening.
      ENUMERATE .o compact :: (spec) (1)
      :: Focus pawns and/or pieces on the central four squares.
      :: Activate the minor pieces.
      :: Castle.
    >>

    LBTEXAMPLE .o vertical
    :: (caption) A custom shopping list
    :: (label) ex:listtable-list-custom
    :: .v <<
      CMD newlist :: shoppinglist :: itemize :: 1
      CMD setlist :: [shoppinglist]
      :: label=\ding{111}, left=4pt, itemsep = 2pt, topsep = 2pt

      TEXT Things to buy on the way home:
      ITEMIZE .o env = shoppinglist
      :: oat milk
      :: tofu
      :: vegan cheese
    >>

    % ----------------------------------------------- Description lists
    SECTION Description lists

    NOTE These are not implemented yet. Watch this space.

    % ----------------------------------------------- Automatic multi-level lists
    SECTION Automatic multi-level lists

    TEXT
    :: If you want to typeset a multi-level list using standard Latex environments, you will end up with a lot of boilerplate. The \lbtlogo{} command \code{LIST} offers great convenience, as \cref{ex:listtable-list-multi} demonstrates.
    :: The markers chosen are not a great fit for the list content, but they show some of the possibilities. The \code{49} refers to the dingbats provided by the \package{pifont} package. See the documentation for more details.

    LBTEXAMPLE .o horizontal
    :: (caption) A mutli-level list
    :: (label) ex:listtable-list-multi
    :: .v <<
      LIST .o markers = circnum * 49
      :: Fruits
      :: * Citrus
      :: * * Orange
      :: * * Lemon
      :: * * Lime
      :: * Berries
      :: * * Strawberry
      :: * * Blueberry
      :: Vegetables
      :: * Leafy greens
      :: * * Spinach
      :: * * Kale
      :: * Root vegetables
      :: * * Carrot
      :: * * Beetroot
      :: * * Potato
    >>

    % ----------------------------------------------- Tables
    SECTION Tables

    STO fn1 :: 1 :: If you really want \lbtlogo{} to support \package{tabularx} or something else, feel free to request it. Of course, you can implement it yourself, too.
    TEXT
    :: \lbtlogo{} makes an opinionated choice that tables are best set using \package{tabularray}.\footnote{◊fn1} The \code{TABLE} command is a light layer over a \code{tblr} environment. You provide all the specifications (details of column alignment and cell formatting) in the mandatory \code{spec} kwarg.
    :: \cref{ex:listtable-table-1} shows left, center and right column alignment, and bold top-row headings, and a horizontal line between the headings and the data. This is a very simple table.
    :: \cref{ex:listtable-table-2} improves on the above by using the \package{booktabs} library to gain access to \latexcmd{toprule}, \latexcmd{midrule} and \latexcmd{bottomrule}. It also introduces some padding to the third column, and increases the row separation.
    :: Most tables created with Latex are for displaying information in articles or books. But tables can be used for other purposes, such as educational handouts. \cref{ex:listtable-table-3} demonstrates the use of space (setting column widths and row heights) and background colour. It also puts most cells in math mode, and shows all borders.
    :: It takes a bit of digging in the \package{tabularray} manual to find the right incantations to make all this happen, but ultimately it is easier to achieve the desired results with this package than with any other. And if you need multiple tables with the same format, it is easy to create your own table type with \latexcmd{NewTblrEnviron}. If you created an \code{invoice} table, for instance, you could invoke it with \code{TABLE .o invoice}.

    NOTE No attempt has been made so far to create tables with spanning cells. Whether any specific support from \lbtlogo{} is required remains to be seen.

    NOTE The thing about \code{TABLE .o invoice} is a bald-faced lie. That needs to be implemented. (Easy, though.)

    SUBSECTION Loading table data from a file

    TEXT The \code{TABLE} command makes it easy to load CSV or TSV data from a file and use it as the rows of the table. \cref{ex:listtable-table-dataload} demonstrates this with cumulative normal distribution values, which are clearly better located in a data file than in a Latex file.

    LBTEXAMPLE .o vertical
    :: (caption) Simple table example
    :: (label) ex:listtable-table-1
    :: .v <<
      TABLE .o center :: (spec) colspec = {l c r},
        » row{1}={font=\bfseries}
      :: Author & Year of Birth & Published Works
      :: \hline
      :: William Shakespeare & 1564 & 39
      :: Jane Austen         & 1775 & 7
      :: Charles Dickens     & 1812 & 20
      :: Leo Tolstoy         & 1828 & 48
    >>

    LBTEXAMPLE .o vertical
    :: (caption) Simple table example with more formatting
    :: (label) ex:listtable-table-2
    :: .v <<
      TABLE .o center :: (spec) colspec = {l c r},
      » row{1}={font=\bfseries}, cell{2-Z}{3}={appto=\hspace*{3em}}, rowsep=3pt
      :: \toprule
      :: Author & Year of Birth & Published Works
      :: \midrule
      :: William Shakespeare & 1564 & 39
      :: Jane Austen         & 1775 & 7
      :: Charles Dickens     & 1812 & 20
      :: Leo Tolstoy         & 1828 & 48
      :: \bottomrule
    >>

    LBTEXAMPLE .o vertical
    :: (caption) Table with spacing, color, math mode and lines
    :: (label) ex:listtable-table-3
    :: .v <<
      TABLE .o center :: (spec) colspec = {cccc}, hlines, vlines, row{2-Z}={7mm, mode=math},
      » column{1-2}={2cm}, column{3-4}={3cm}, row{1}={bg=Turquoise!35, font=\bfseries}
      :: Sum & Product
      :: 11 & 18 & 9+2=11 & 9\times2=18
      :: 21 & 110 & 10+11=21 & 10\times11=110
      :: 12 & 36
      :: 18 & 77
      :: 16 & 48
      :: 13 & 40
    >>

    LBTEXAMPLE .o vertical, scale = 0.9
    :: (caption) Table with data loaded from a file
    :: (label) ex:listtable-table-dataload
    :: .v <<
      TABLE .o centre
      :: (spec) colspec={rrrrrrrrrrr}, hlines, vlines,
      »    column{1}={bg=gray9, font=\bfseries}, row{1} = {bg=gray9, font=\itshape},
      »    cell{1}{1} = {mode=math}
      :: (datafile) media/standard-normal-cumulative.tsv
      :: @datarows 1
      :: @datarows 22..27
    >>

\end{lbt}

\begin{lbt}
  !NODRAFT
  [@META]
    TEMPLATE lbt.Doc.Chapter
    TITLE Mathematical text \textsf{(lbt.Math)} -- various macros
    LABEL ch.math-macros
    SOURCES LbtDoc

  [+BODY]
    STO simplemath :: 999 :: \code{simplemath}

    TEXT The \package{lbt.Math} template provides several affordances for typing mathematical text.
    ITEMIZE .o compact
    :: The ◊simplemath macro is a replacement for the inline and display math environments \Verb|$ ... $| and \Verb|$$ .. $$| (or their Latex equivalents). It allows you to type mathematical text more succinctly and with fewer backslashes.
    :: The \code{integal} macro simplifies typing definite and indefinite (single) integrals.
    :: The \code{vector} and \code{vectorijk} macro greatly simplify typing vectors.
    :: A collection of macros like \code{mathlistand}, \code{mathsum}, \code{mathseqdots} and several others provide a convenient way to type mathematical text like \qq{\mathlistand{a,b,c}} or \qq{\mathseqdots{y,1,2,3,n}}.

    TEXT Also found in \package{lbt.Math} is the \code{MATH} command, which aids in the typesetting of display equations. It appears in \cref{ch.math-command}.

    % ------------------------------------------------------------ The simplemath macro
    SECTION The ◊simplemath macro

    BOX :: \verb|\lbtDefineMacros{sm = lbt.Math:simplemath}|

    TEXT
    :: The ◊simplemath macro stands in for \Verb|$ ... $| or \Verb|$$ .. $$| and allows your to type mathematical text succinctly. It recognises a lot of keywords and abbreviations, meaning far fewer backslashes are needed.
    :: \cref{ex:math-sm-1} demonstrates the use of ◊simplemath in both math modes: inline and display. \cref{ex:math-sm-table} contains a large number of examples to show the variety of conveniences that ◊simplemath offers.
    :: \cref{ex:math-sm-2} shows that brackets, both round and square, are auto-sized (that is, \Verb|\left| and \Verb|\right| are applied intelligently) and that \Verb|text{...}| (or \Verb|\text{...}|) is recognised specially, and passed through to Latex without processing its contents. Note, however, that braces (\Verb|{}|) are \emph{not} auto-sized.

    LBTEXAMPLE .o vertical
    :: (caption) ◊simplemath usage, both inline and display form
    :: (label) ex:math-sm-1
    :: (breakat) BREAK
    :: .v <<
      TEXT Two interesting results from Euler are BREAK \sm{ds arctan x = x - frac{x^3}3 + frac{x^5}5 - cdots} and BREAK \sm{ sum_{n=1}^{infty} frac 1 {n^2} = frac {pi^2} 6. }
    >>

    EXAMPLETABLE .o wrapcell = sm
    :: (label) ex:math-sm-table
    :: (caption) A collection of ◊simplemath examples
    :: cos2 th + sin2 th equiv 1
    :: log_2 n ge 5
    :: ds (1+x)^n = sum_{r=0}^n binom n r x^r
    :: A = B iff A subseteq B vee B subseteq A
    :: ds al be + al ga + be ga = frac {-b} {2a}
    :: ds f'(x) = lim_{h to 0} frac {f(x+h)-f(x)} h
    :: neg(P implies Q) equiv P vee neg Q
    :: P imp Q equiv neg P wedge Q
    :: 12 = 2 cdot 2 cdot 3 text{ so } 5 nmid 12
    :: ds prod_{i=1}^n x_i = x_1 x_2 cdots x_n
    :: sqrt 2 notin bbQ
    :: forall n in bbZ, n^2 in bbZ
    :: OABC text{ is a parallelogram.}


    LBTEXAMPLE .o vertical
    :: (label) ex:math-sm-2
    :: (caption) ◊simplemath's handling of parentheses, brackets and \code{text}
    :: (breakat) BREAK
    :: .v <<
      TEXT Brackets will automatically assume the correct size: BREAK
      » \sm{ y = (1 + [frac x 7])^2 quad text{($[A]$ is the rounding function)} }

      TEXT Suppress this behaviour by setting the option \texttt{simplemath.leftright = false}.
    >>

    % ------------------------------------------------------------ The integral macro
    LATEX \FloatBarrier
    SECTION The \code{integral} macro

    TEXT
    :: Typing integrals in regular Latex, or with ◊simplemath, is not exactly a chore. But there is room for simplification, and the \code{integral} macro allows for definite or easy typing indefinite (single, simple) integrals. If the first argument is \code{ds} then a \Verb|\displaystyle| is inserted.
    :: The resulting Latex code is wrapped in \Verb|\ensuremath{...}|, so integrals do not need to be inside a math environment.
    :: \cref{ex:math-int-1} shows a definite and indefinite integral in context. \cref{ex:math-int-2} contains enough examples so that definite and indefinte are shown, as are normal style and display style.

    LBTEXAMPLE .o vertical
    :: (caption) The \code{integral} macro
    :: (label) ex:math-int-1
    :: (breakat) BREAK
    :: .v <<
      TEXT The integral \integral{ds,\frac{\sin x}{x},dx} has practical importance but can't be evaluated in closed form.

      TEXT You can take advantage of ◊simplemath as well: BREAK \sm{ \integral{pi/3,pi,sqrt {1 + sin3 th},d th} }
    >>

    EXAMPLETABLE
    :: (label) ex:math-int-2
    :: (caption) A collection of \code{integral} examples
    :: \integral{\sin x,dx}
    :: \integral{1,3,\sqrt{4z},dz}
    :: \integral{ds,\tan y,dy}
    :: \integral{ds,1,3,\sqrt{4z},dz}



    % ------------------------------------------------------------ The vector macro and friends
    LATEX \FloatBarrier
    SECTION The \code{vector} macro

    TEXT
    :: With the \code{vector} macro you can easily typeset \vecbold{a} and \vectilde{b} and \vecarrow{c}. Oh, and \V{DE} and \V{3 -7} and \V{row 4 0 9}. And finally, \V{ijk 2 -5 1} and \V{ijk -3 0 4}.
    :: One might choose to set up this macro as \Verb|\V| as follows:
    BOX \verb|\lbtDefineMacros{V = lbt.Math:vector}|
    TEXT A thorough set of examples is given in \cref{ex:math-vector-table}.

    EXAMPLETABLE .o mathmode
    :: (label) ex:math-vector-table
    :: (caption) A collection of \code{vector} and \code{vectorijk} examples
    :: \V{3 1 -9} + \V{-2 5 4} = \V{1 6 -5}
    :: \V{row 2 7 -1 4 6}
    :: \V{a} + \V{b} = \V{a_1 a_2} + \V{b_1 b_2}
    :: \V{r} = \V{r_1 \vdots r_n}
    :: \V{3 1 -9} = \Vijk{3 1 -9}
    :: \V{-1 0 4} = \Vijk{-1 0 4}
    :: \V{-1 0} = \Vijk{-1 0}
    :: \V{ijk 3 4 5}

    SUBSECTION Bold, tilde and arrow
    TEXT Vectors are most commonly set in bold upright (\vecbold{a}), and that is the \lbtlogo{} default. You can, however, choose tilde (\vectilde{a}) or arrow (\vecarrow{a}) instead. \cref{ex:math-vector-formats-1} shows how to make these choices for your \lbtlogo{} expansion or your whole Latex document. If you want just a one-off, you can define and use the following macros. \cref{ex:math-vector-formats-2} demonstrates their use.
    BOX
    :: \verb|\lbtDefineMacros{vecbold  = lbt.Math:vecbold}|
    :: \verb|\lbtDefineMacros{vectilde = lbt.Math:vectilde}|
    :: \verb|\lbtDefineMacros{vecarrow = lbt.Math:vecarrow}|

    LBTEXAMPLE .o output = 0
    :: (caption) Vectors in bold or tilde or arrow format, document-wide
    :: (label) ex:math-vector-formats-1
    :: (substitute) LBT/lbt
    :: .v <<
      * Affect the whole Latex document
      \lbtGlobalOpargs{vector.format = tilde}

      * Affect one LBT expansion
      \begin{LBT}
        [@META]
          TEMPLATE lbt.Basic
          OPTIONS  vector.format = arrow
        ...
      \end{LBT}
    >>

    LBTEXAMPLE .o vertical
    :: (caption) Vectors in bold or tilde or arrow format, one-off
    :: (label) ex:math-vector-formats-2
    :: .v <<
      TEXT We can write \vecbold{a} or \vectilde{b} or \vecarrow{c}.
    >>

    % ---------------------------------------------------- The mathlistand macro and friends
    LATEX \FloatBarrier
    SECTION The \code{mathlistand} macro and friends

    % ------------------------------------------------------------ Code to set up all macros
    LATEX \FloatBarrier
    SECTION Code to set up all macros

    TEXT
    :: As a convenience, \cref{ex:math-macros-setup} contains the code that you can paste into your preamble to obtain access to all macros described in this chapter. They are grouped for readability.

    LBTEXAMPLE .o output = 0
    :: (label) ex:math-macros-setup
    :: (caption) Preamble code to set up all \package{lbt.Math} macros
    :: .v <<
      \lbtDefineLatexMacros{
        sm                 = lbt.Math:simplemath,
        integral           = lbt.Math:integral,
      }
      \lbtDefineLatexMacros{
        V                  = lbt.Math:vector,
        vecbold            = lbt.Math:vecbold,
        vecarrow           = lbt.Math:vecarrow,
        vectilde           = lbt.Math:vectilde,
      }
      \lbtDefineLatexMacros{
        mathlistand        = lbt.Math:mathlistand,
        mathlistor         = lbt.Math:mathlistor,
        mathlist           = lbt.Math:mathlist,
        mathlistdots       = lbt.Math:mathlistdots,
        mathsum            = lbt.Math:mathsum,
        mathsumdots        = lbt.Math:mathsumdots,
        mathseq            = lbt.Math:mathseq,
        mathseqsum         = lbt.Math:mathseqsum,
        mathseqproduct     = lbt.Math:mathseqproduct,
        mathseqdots        = lbt.Math:mathseqdots,
        mathseqdotssum     = lbt.Math:mathseqdotssum,
        mathseqdotsproduct = lbt.Math:mathseqdotsproduct,
      }
      \lbtDefineLatexMacros{
        primefactorisation = lbt.Math:primefactorisation,
      }
    >>




\end{lbt}

\begin{lbt}
  !DRAFT
  [@META]
    TEMPLATE lbt.Doc.Chapter
    TITLE Mathematical text \textsf{(lbt.Math)} -- the \code{MATH} command
    LABEL ch.math-command
    SOURCES LbtDoc

  [+BODY]
    STO lbt :: 999 :: \lbtlogo{}
    STO MATH :: 999 :: \code{MATH}
    STO amsmath :: 999 :: \package{amsmath}
    STO mathtools :: 999 :: \package{amsmath}
    STO split :: 999 :: \code{split}
    STO eqsplit :: 999 :: \code{eqsplit}
    STO equation :: 999 :: \code{equation}
    STO align :: 999 :: \code{align}
    STO gather :: 999 :: \code{gather}
    STO multline :: 999 :: \code{multline}
    STO simplemath :: 999 :: \code{simplemath}
    STO xxx :: 999 :: \code{xxx}
    STO xxx :: 999 :: \code{xxx}

    TEXT
    :: The ◊MATH command gets its own chapter so that its various features can be displayed one section at a time.
    :: ◊MATH provides for a variety of display equations. It is a portal to various \package{amsmath} and \package{mathtools} environments like ◊split, ◊gather, ◊align, and so on. The examples here give a good primer on their use, but readers should consult the relevant documentation to develop greater awareness of the details.

    % ------------------------------------------------------------ Opening remarks
    SECTION Opening remarks

    STO fn1 :: 1 :: Or the Tex command \Verb|$$ ... $$|, which is lower-level and may produce different vertical spacing from \Verb|\[ ... \]|.

    TEXT
    :: Setting a display equation with \Verb|\[ ... \]|\footnote{◊fn1} is enough for a great many cases. If you want your equation to be numbered, you upgrade to the \code{equation} environment. If the math content to be displayed is more complicated than that, the author should decide which of the following applies:
    ITEMIZE .o compact
    :: there is one logical equation with several parts (separated by $=$ or $>$ or \dots) that should appear on separate lines;
    :: sthere is one logical equation that is too long to fit on one line;
    :: there are several logical equations to be displayed together, centered or left-aligned;
    :: there are several logical equations to be displayed reasonably simple alignment;
    :: there are more complicated alignment requirements, perhaps involving comments to the side
    TEXT
    :: Based on that, the author can choose an ◊amsmath environment, as summarised in \cref{table:amsmath-envs}. The table does not show \emph{all} available environments, but it gives a good overview for readers who are not already familiar.
    :: The sections of this chapter give more detailed information on these environments and more.

    % NEWCOMMAND amsexample :: 2 :: \begin{minipage}{7cm}{\begin{#1} #2 \end{#1}}
    NEWCOMMAND amsexample :: 2 :: \begin{minipage}{\linewidth}{\setcounter{equation}{0}\begin{#1} #2 \end{#1}}\end{minipage}

    STO splitexample :: 1 :: \begin{split}f'(x) &= \lim_{h\to0} \frac{f(x+h) - f(x)}{h} \\ &= \lim_{h\to0} \frac{(x+h)^2 - x^2}{h} \\ &= \lim_{h\to0} \frac{(x^2 + 2xh + h^2) - x^2}{h} \\ &= \lim_{h\to0} \frac{2xh + h^2}{h} \\ &= \lim_{h\to0} 2x + h \\ &= 2x\end{split}
    STO half :: 1 :: \frac{1}{2}

    TABLE .o float
    :: (spec) colspec = {lX}, cells = {font=\small}, row{1} = {font=\bfseries\small}, cell{1}{2} = {halign=c}
    :: (caption) Some environments provided by ◊amsmath
    :: (label) table:amsmath-envs
    :: Environment & Example
    :: \hline
    :: ◊equation & \amsexample{equation}{a^2 + b^2 = c^2}
    :: ◊gather   & \amsexample{gather}{a^2 + b^2 = c^2 \\ E = mc^2 \\ F = k \frac{q_1q_2}{r^2}}
    :: ◊align (1)   & \amsexample{align}{a^2 + b^2 &= c^2 \\ E &= mc^2 \\ F &= k \frac{q_1q_2}{r^2}}
    :: ◊align (2)    & \amsexample{align}{a^2 + b^2 &= c^2  &  E &= mc^2 \\ F &= k \frac{q_1q_2}{r^2}  &  F &= ma}
    :: ◊align (3)    & \amsexample{align}{2^{n+1} &= 2\cdot 2^n \\ &> 2\cdot n^2 &&\text{by assumption} \\ &= n^2 + ◊half n^2 + ◊half n^2 \\ &> n^2 + 2n + 1 &&\text{reader to confirm} \\ &= (n+1)^2}
    :: ◊split {\footnotesize(inside ◊equation)} & \amsexample{equation}{◊splitexample}
    :: ◊multline & \amsexample{multline}{(1+x)^n = \sum_{r=0}^n \binom{n}{r}\,x^r = 1 + \binom n1 x + \binom n2 x^2 \\ + \dots + \binom nr x^r + \dots + \binom nn x^n}

    % ------------------------------------------------------------ equation
    SECTION ◊equation

    TEXT
    :: The ◊equation environment provides for a simple numbered equation. \cref{ex:math-equation} demonstrates this in ◊lbt.
    :: Equation numbering is suppressed by default (◊lbt author's preference), but is enabled with the \code{eqnum} oparg. If you want equation numbering enabled by default, you can set \code{MATH.eqnum = true} as a local or global oparg. See \textbf{Section ??? -- to be written}. Then you can specify \code{.o noeqnum} for a particular equation if you wish.
    :: This aspect of equation numbering -- off by default -- applies to all environments shown in this chapter.

    LBTEXAMPLE .o horizontal, reseteqnum
    :: (label) ex:math-equation
    :: (caption) \code{MATH .o equation} to format a simple equation
    :: .v <<
      TEXT Newton's second law is known to many.
      MATH .o equation :: F = ma

      TEXT You can specify that the equation should be numbered.
      MATH .o equation, eqnum :: F = ma

      TEXT Finally, equation is the \emph{default} environment for \code{MATH}, so you can simply write:
      MATH F = ma
    >>

    % ------------------------------------------------------------ eqsplit
    LATEX \FloatBarrier
    SECTION ◊eqsplit

    SUBSECTION ◊eqsplit
    TEXT
    :: The ◊amsmath environment ◊split is designed for a single logical equation that is broken into two or more lines, like the example below.
    % MATH .o eqsplit                 % XXX: eqsplit is not yet implemented!
    %   :: (a+b)^2 &= (a+b)(a+b)
    %   ::         &= a^2 + ab + ab + b^2
    %   ::         &= a^2 + 2ab + b^2
    TEXT
    :: However, ◊split is a
    :: It must occur within another math environment, like ◊equation or ◊align, or simply \Verb|\[ ... \]|. That is because you may have several such equations grouped in the enclosing environment. A split equation gets only one number, not one number per line, and the numbering comes from the enclosing environment.
    :: \textbf{The same thing is said three times below. Time to edit!}
    :: It is often inconvenient not to be able to treat ◊split as a top-level environment, so \code{MATH} provides \code{eqsplit}, which is ◊split wrapped in ◊equation. \cref{ex:math-eqsplit} demonstrates this feature.
    :: For convenience, \code{MATH} provides ◊eqsplit, which is a ◊split environment wrapped in an ◊equation environment.
    :: If you want to show an equation split across several lines, chances are good it's the \emph{only} equation you want to show, which means in ordinary Latex you have an ◊equation environment containing a ◊split environment and nothing else. To handle this common case, ◊lbt provides ◊eqsplit, demonstrated in \cref{ex:math-eqsplit}. This example also shows that ◊simplemath can be used if you specify the \code{sm} oparg.

    LBTEXAMPLE .o vertical
    :: (label) ex:math-eqsplit
    :: (caption) Using \code{MATH .o eqsplit} for a multi-step equation
    :: .v <<
      TEXT Part of a proof by induction.
      STO half :: 1 :: $\tfrac 1 2$
      MATH .o split, sm
      :: 2^{n+1} &= 2 cdot 2^n
      ::         &> 2 cdot n^2
      ::         &= n^2 + ◊half n^2 + ◊half n^2
      ::         &> n^2 + 2n + 1
      ::         &= (n+1)^2

      TEXT A numbered split equation.
      MATH .o split, eqnum, sm, label = eq1
      :: (a+b)^2 &= (a+b)(a+b)
      ::         &= a^2 + ab + ab + b^2
      ::         &= a^2 + 2ab + b^2

      TEXT As shown in \cref{eq1}, $(a+b)^2$ does not equal $a^2 + b^2$!
    >>


    % ------------------------------------------------------------ align
    LATEX \FloatBarrier
    SECTION \code{align}

    % ------------------------------------------------------------ gather
    LATEX \FloatBarrier
    SECTION \code{gather}



    % ------------------------------------------------------------ multiline
    LATEX \FloatBarrier
    SECTION \code{multiline}



    % ------------------------------------------------------------ Other environments
    LATEX \FloatBarrier
    SECTION (label) sec:math-other :: \code{Other environments}

    SUBSECTION ◊split
    TEXT
    :: \textbf{Revisit this text in light of it being in the \qq{other} section}
    :: ◊MATH provides the \code{split} option to access the ◊split environment, but it is not likely to be all that useful, because of the need to enclose it in another environment. The example below shows the ◊lbt code and resulting Latex code.

    LBTEXAMPLE .o float = false, horizontal, output = 1, shrinkmargin = nil
    :: .v <<
      MATH .o split
      :: (a+b)^2 &= (a+b)(a+b)
      ::         &= a^2 + ab + ab + b^2
      ::         &= a^2 + 2ab + b^2
    >>




    % ------------------------------------------------------------ xxx
    LATEX \FloatBarrier
    SECTION \code{xxx}



    % ------------------------------------------------------------ Summary
    LATEX \FloatBarrier
    SECTION \code{Summary of the \code{MATH} command}





\end{lbt}


\part{Non-core built-in templates}


\begin{lbt}
  [@META]
    TEMPLATE  lbt.Doc.Chapter
    TITLE     Worksheet or exam questions with lbt.Questions
    LABEL     ch.questions
    SOURCES   LbtDoc, lbt.Questions

  [+BODY]

    TEXT The \package{lbt.Questions} template offers useful commands for typesetting questions, subquestions, and multiple-choice options.

    TEXT Use \code{Q} for a top-level question and \code{QQ} for a question part (see \cref{ex:questions}).

    LBTEXAMPLE .o horizontal
    :: (caption) Question parts and subparts
    :: (label) ex:questions
    :: .v <<
      Q Name three different kinds of clouds.

      Q Evaluate the following.
      QQ $3 + 12 / 4$
      QQ $(3 + 12) / 4$
    >>

    TEXT \cref{ex:q-hor} shows the use of \code{QQ*} to lay out questions parts horizontally.

    LBTEXAMPLE .o vertical
    :: (caption) Arranging question parts horizontally
    :: (label) ex:q-hor
    :: .v <<
      Q How many vowels appear in each word?
      QQ* [ncols=3]
      :: appear :: Augustine :: crimson :: toast :: glyph :: transformer
    >>

    TEXT Use \code{MC} or \code{MC*} to lay out \textbf{multiple-choice options} (see \cref{ex:q-mc}).

    LBTEXAMPLE .o vertical
    :: (caption) Multiple choice answers
    :: (label) ex:q-mc
    :: .v <<
      Q Which planet of the solar system has the most moons?
      MC Earth :: Mars :: Jupiter :: Saturn

      Q Which planet of the solar system has the fewest moons?
      MC* [ncols=4] :: Mercury :: Venus :: Uranus :: Neptune
    >>

    TEXT A question can have a \textbf{source} and/or a \textbf{note} preceding the text, and you can change the colour of the question marker (see \cref{ex:q-source-notes}).

    LBTEXAMPLE .o vertical
    :: (caption) Question source and notes
    :: (label) ex:q-source-notes
    :: .v <<
      Q .o color=purple :: (source) HSC 2005 :: (note) sigma notation
      :: Evaluate $\displaystyle \sum_{n=3}^5 (2n+1)$.
    >>

    TEXT If you want \emph{all} questions in your document to have a purple marker, you can include the line \texttt{OPTIONS  Q.color = purple} in the \texttt{[@META]} part of your \lbtlogo{} environment.

\end{lbt}


\part{Creating a new template}

\part{Extra features}

% -------------------------------------------------- Back matter
\appendix

\backmatter
% bibliography, colophon, index, etc.

\end{document}
